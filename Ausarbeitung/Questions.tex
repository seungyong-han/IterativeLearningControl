\chapter{Questions} 

\section{Session 30.02.2020}
System we consider 
\begin{align}
e_{k+1} = L(\Delta)e_k = \Delta \star L e_k.
\end{align}

$L$ is already a closed loop, i.e. $L = N$ from RC. 
Thm 6.19: If $\mu_{\mathbb{V}_e}(L) \leq 1 $, then RS and RP are guaranteed for all $\Delta \in \Delf$ with $||\Delta|| < 1$ (if well-defined). 

%
%
%$P$ without uncertainty: 
%\begin{align}
%\left[
%\begin{array}{c}
%u_{k+1}\\ \hline y_k \\ \hline  e_k 
%\end{array} \right] &= 
%\left[\begin{array}{c|c|cc}
%I & I & 0 & 0 \\
%\hline
%G & 0 & 0 & I\\
%\hline
%-G & 0 & I & -I
%\end{array}\right]
%\left[\begin{array}{c}
%u_k \\ \hline v_k \\ \hline r \\ d 
%\end{array}\right]\\
%v_k &= K e_k = K(r - Gu_k - d) = -KGu_k + Kr - Kd
%\end{align}
%
%%$P \star K $ without uncertainty:
%%\begin{align}
%%\left[\begin{array}{c}
%%u_{k+1} \\ 
%%\hline e_k
%%\end{array}\right] = 
%%\left[\begin{array}{c | c c}
%%(I-KG) & K & -K \\
%%\hline
%% -G & I & -I
%%\end{array}\right]
%%\left[\begin{array}{c}
%%u_{k} \\ \hline  r \\ d
%%\end{array}\right]
%%\end{align}
%
%$P\star K$ with uncertainty: 
%\begin{align}
%\left[\begin{array}{c}
%u_{k+1} \\\hline e_k \\ \hline z_k
%\end{array}\right] &= 
%\left[\begin{array}{c| c| c}
%A & B_1 & B_2\\
%\hline
%C_1 & D_{11} & D_{12} \\
%\hline
%C_2 & D_{21} & D_{22}
%\end{array}\right]
%\left[\begin{array}{c}
%u_k \\ \hline r \\ d   \\\hline w_k
%\end{array}\right],\\
%w_k &= \Delta (\delta) z_k.
%\end{align}
%
%
%
%\textbf{Matrix seen by uncertainty}: $M = D_{22}$. 
%\\
%
\textbf{Exercise}: $(I - D_{22} \Delta)^{-1}$ exists and is \textbf{stable} if there exists a matrix $P = P^T$, such that 
\begin{align}
\label{ex} 
\begin{bmatrix}
I\\
\Delta(\delta)
\end{bmatrix}^T 
P
\begin{bmatrix}
I \\ \Delta(\delta)
\end{bmatrix} \succeq 0,
\text{ and }
\begin{bmatrix}
D_{22} \\ I 
\end{bmatrix}^T
P
\begin{bmatrix}
D_{22} \\ I
\end{bmatrix} \prec 0. \tag{*}
\end{align}

For real parametric uncertainty with one parameter $\Delta(\delta) = \delta I$: \eqref{ex} $\Rightarrow$ $(I - D_{22}\delta I )^{-1}$ exists and is  \textbf{stable}. 


\begin{proof} $M = D_{22}$. 
	Consider $\{\delta I: \delta \in [-1,1]\} \subset \Delta\!\!\!\!\Delta = \{\delta I : \delta \in \C, |\delta| < 1\}$.
	Suppose there exists some $\delta_0 \in \C$, $|\delta_0| < 1$, such that there exists some $\lambda \in \C$, $|\lambda|>1$, $\lambda \in \eig(I - M\delta_0)$. 
	\begin{align*}
	\Rightarrow& \exists x\neq 0: \; (I - M \delta_0 ) x = \lambda x \Rightarrow M \delta_0 x = (1-  \lambda) x 
	\Rightarrow \frac{1-\lambda}{\delta_0} \in \eig(M). \\
	\text{ If }& \left|\frac{1- \lambda}{\delta_0}\right| \geq 1 \Rightarrow M \text{ is unstable } \text{\Lightning}.\\
	\text{ If }& \left|\frac{1 - \lambda}{\delta_0}\right| < 1 \Rightarrow \frac{1 - \lambda}{\delta_0}I \in \Delta\!\!\!\!\Delta \Rightarrow \frac{-1 +\lambda}{\delta_0}I \in \Delta\!\!\!\!\Delta \Rightarrow \det\left(I + M \frac{-1 + \lambda}{\delta_0}\right) \neq 0 \Rightarrow \frac{1 - \lambda}{\delta_0} \notin \eig(M) \text{\Lightning}.
	\end{align*}
\end{proof}
%
%
%For real parametric uncertainty $\Delta(\delta) = \delta I$, $\delta \in [-1, \, 1],$
%\begin{align}
%\mathbb{P} = \{P = \begin{bmatrix}
%Q & S \\
%S^T & -Q
%\end{bmatrix}: Q \preceq 0, S = -S^T \},.
%\end{align}
%
%
%For $\Delta = \begin{bmatrix}
%\delta_1 I & & & \\
%& \delta_2 I & & \\
%& & \ddots & \\
%& & & \delta_\tau I
%\end{bmatrix}$:
%\begin{align}
%\mathbb{P} = \left\{\begin{array}{c|c}
% P = \left[
% \begin{array}{c| c c c c}
%P_0 & & & &\\
%\hline
%& P_1 & & & \\
%& & P_2 & & \\
%& & & \ddots &\\
%& & & & P_\tau
%\end{array}
%\right] &
%P_i \succeq 0
%\end{array}
%\right\}
%\end{align}
%
%
%\par\noindent\rule{\textwidth}{0.4pt}
%\section{From last session:}
%
%\begin{align}
%e_{k+1} = L(\delta) e_k.
%\end{align}
%Need:
%\begin{align}
%\label{sess}
%||L(\delta)||<1 \Leftrightarrow L(\delta)^T L(\delta) < I \Leftrightarrow 
%\begin{bmatrix}
%I \\ L(\delta)
%\end{bmatrix}^T 
%\begin{bmatrix}
%-I  & 0 \\ 0 & I
%\end{bmatrix}
%\begin{bmatrix}
%I \\ L(\delta)
%\end{bmatrix}<0 \; \forall \delta \in [-1,\,1]. \tag{$\circ$}
%\end{align}
%
%\begin{align*}
%\begin{bmatrix}
%e_{k+1} \\ z_k 
%\end{bmatrix} &= 
%\left[\begin{array}{c | c}
%\c{A} & \c{B} \\\hline
%\c{C} & \c{D}
%\end{array}\right] 
%\begin{bmatrix}
%e_k \\ w_k
%\end{bmatrix}\\
%L(\delta) &= \Delta(\delta)\star \left[\begin{array}{c | c}
%\c{A} & \c{B} \\\hline
%\c{C} & \c{D}
%\end{array}\right]
%\end{align*}
%
%
%
%
%
%\begin{align}
%\label{qeq1}
%\eqref{sess} \Leftarrow \exists P = P^T: 
%\begin{bmatrix}
%I & 0 \\ \c{A} &  \c{B}
%\end{bmatrix}^T\begin{bmatrix}
%-I & 0 \\ 0 & I
%\end{bmatrix} 
%\begin{bmatrix}
%I & 0 \\ \c{A} &  \c{B}
%\end{bmatrix} + 
%\begin{bmatrix}
%\c{C} & \c{D} \\ 0 & I
%\end{bmatrix}^T
%P
%\begin{bmatrix}
%\c{C} & \c{D} \\ 0 & I
%\end{bmatrix} \prec 0, \\
%\begin{bmatrix}
%I \\ \Delta(\delta) 
%\end{bmatrix}^T P \begin{bmatrix}
%I \\ \Delta(\delta) 
%\end{bmatrix} \succeq 0 \; \forall \delta \in [-1,\, 1]. \nonumber 
%\end{align}
%
%For  $P = \begin{bmatrix}
%P_1 & 0 \\ 0 & P_2
%\end{bmatrix}, P_1 = P_1^T, P_2 = P_2^T$
%\begin{align*}
%\eqref{qeq1}
%\Leftrightarrow \begin{bmatrix}
%- I + \c{A}^T \c{A} & \c{A}^T \c{B} \\ \c{B}^T \c{A} & \c{B}^T \c{B}
%\end{bmatrix} + 
%\begin{bmatrix}
%\c{C}^T P_1 \c{C} & \c{C}^T P_1 \c{D} \\ \c{D}^T P_1 \c{C} & \c{D}^T P_1 \c{D} + P_2  
%\end{bmatrix}\prec 0,
%\begin{bmatrix}
%I \\ \Delta(\delta) 
%\end{bmatrix}^T P \begin{bmatrix}
%I \\ \Delta(\delta) 
%\end{bmatrix} \succeq 0 \; \forall \delta \in [-1,\, 1].
%\end{align*}
%
%
%Discrete Lyapunov inequation: 
%\begin{align*}
%L^T X L - X \prec 0 \Leftrightarrow 
%\begin{bmatrix}
%I \\ L(\delta) 
%\end{bmatrix}^T \begin{bmatrix}
%-{\color{dgreen}X}& 0 \\ 0 &{\color{dgreen}X}
%\end{bmatrix}
%\begin{bmatrix}
%I \\ L(\delta) 
%\end{bmatrix} \prec 0, \text{ for } {\color{dgreen} X} \succ 0, \\
%\Leftrightarrow 
%\c{A}^T{\color{dgreen}X}\c{A} + \c{C}^T (I - \c{D} \Delta(\delta))^{-T} \c{B}^T \Delta(\delta) {\color{dgreen}X}\c{A} + 
% \c{A}^T{\color{dgreen}X} \c{B} \Delta(\delta) ( I - \c{D}\Delta)^{-1}\c{C} +\\+  \c{C}^T (I - \c{D} \Delta(\delta))^{-T} \c{B}^T \Delta(\delta) {\color{dgreen}X} \c{B} \Delta(\delta) ( I - \c{D}\Delta)^{-1}\c{C} +  \c{C}^T (I - \c{D}) - {\color{dgreen} X}\prec 0
%\end{align*}
%
%\begin{align*}
%L(\delta) ^T L(\delta) - I = \c{A}^T\c{A} + \c{C}^T (I - \c{D} \Delta(\delta))^{-T} \c{B}^T \Delta(\delta) \c{A} + 
%\c{A}^T \c{B} \Delta(\delta) ( I - \c{D}\Delta)^{-1}\c{C} +\\+  \c{C}^T (I - \c{D} \Delta(\delta))^{-T} \c{B}^T \Delta(\delta)  \c{B} \Delta(\delta) ( I - \c{D}\Delta)^{-1}\c{C} +  \c{C}^T (I - \c{D}) - I \stackrel{!}{\prec} 0.
%\end{align*}
%
%
%
%






